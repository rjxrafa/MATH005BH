\documentclass[12pt]{article}
\usepackage{amsmath}
\usepackage{amssymb}
\usepackage{amsthm}
\usepackage{amsfonts}
\usepackage{graphicx} 
\usepackage{amsthm}
\graphicspath{{./images/}}

\title{Desmos \#5:
\\Using the Alternating Series
\\Estimation Theorem}
\author{Rafael Betita\\
MATH 005BH - Single Variable Calculus II}
\date{November 8, 2018}

\begin{document}
\maketitle
\newpage
\section{Review of Absolute & Conditional Convergence}
In sections 11.5 and 11.6, you learned about absolute and conditional convergence. So when we are looking at a series of the form $\sum_{n=1}^\infty(-1)^nb_n$ what is the first thing we consider? Briefly explain, when you look at this series, how you approach checking this series for convergence. What do you do first? Second?

\begin{gather*}
    
\end{gather*}


\end{document}